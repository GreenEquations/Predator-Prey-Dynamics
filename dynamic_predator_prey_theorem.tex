\documentclass[12pt]{article}
\usepackage{amsmath, amssymb, amsthm}
\usepackage{geometry}
\usepackage{graphicx}
\usepackage{hyperref}
\geometry{margin=1in}
\title{Predator-Prey Dynamics and Dynamic Carrying Capacity: Non-linear Approaches to Ecological Stability}
\author{}
\date{}

% Theorem environment
\newtheorem{theorem}{Theorem}

\begin{document}

\maketitle

\section*{Overview}

This theorem extends the traditional Lotka-Volterra predator-prey model by introducing a \textbf{dynamically fluctuating carrying capacity}. It proposes that environmental variability alters equilibrium population sizes, stabilizing predator and prey populations below the fixed carrying capacity assumed in classical models.

\begin{theorem}[Predator-Prey Dynamics and Dynamic Carrying Capacity]
Fluctuations in carrying capacity \( K(t) \) due to environmental and resource changes drive non-linear dynamics in predator-prey systems, leading to dynamic equilibrium points. Predator and prey populations stabilize below the traditionally fixed carrying capacity in response to variability in available resources.
\end{theorem}

\section*{Mathematical Formulation}

\subsection*{1. Dynamic Carrying Capacity}

\[
K(t) = K_0 \left( 1 + \lambda \cdot \sin(\omega t) + \varphi \cdot \cos(\omega t) \right)
\]

Where:
\begin{itemize}
  \item \( K_0 \): Baseline carrying capacity
  \item \( \lambda, \varphi \): Amplitude coefficients
  \item \( \omega \): Frequency of environmental oscillations
\end{itemize}

\subsection*{2. Modified Lotka-Volterra Equations}

\textbf{Prey Dynamics:}
\[
\frac{dP}{dt} = \alpha P \left(1 - \frac{P}{K(t)} \right) - \beta P Q
\]

\textbf{Predator Dynamics:}
\[
\frac{dQ}{dt} = \delta P Q - \gamma Q
\]

Where:
\begin{itemize}
  \item \( P \): Prey population
  \item \( Q \): Predator population
  \item \( \alpha \): Prey growth rate
  \item \( \beta \): Predator-prey interaction coefficient
  \item \( \delta \): Predator reproduction rate per prey
  \item \( \gamma \): Predator death rate
\end{itemize}

\subsection*{3. Equilibrium Populations}

\textbf{Time-dependent equilibrium values:}
\[
P^*(t) = \frac{\alpha K(t)}{\beta + \gamma}, \quad
Q^*(t) = \frac{\delta \alpha K(t)}{\beta(\gamma + \delta)}
\]

These equilibria vary over time and remain below the fixed \( K_0 \), reflecting ecological response to environmental variation.

\section*{Testable Predictions}

\begin{enumerate}
  \item In ecosystems with fluctuating resources (e.g., seasonal or climate-driven), predator and prey populations will stabilize below the fixed carrying capacity, following \( K(t) \).
  \item A dynamic carrying capacity alters the amplitude and frequency of predator-prey oscillations compared to classical models.
  \item Empirical observations from marine or terrestrial ecosystems with resource variability will support the predictions of this model over static-capacity models.
\end{enumerate}

\section*{Peer Review Considerations}

\textbf{Logical Consistency:}
\begin{itemize}
  \item Is the use of fluctuating \( K(t) \) justified biologically?
  \item Could other models explain the same behavior?
\end{itemize}

\textbf{Mathematical Soundness:}
\begin{itemize}
  \item Are the modified equations solvable and stable?
  \item How sensitive is the model to the parameters \( \lambda, \varphi, \omega \)?
\end{itemize}

\textbf{Empirical Feasibility:}
\begin{itemize}
  \item Can this model be validated using ecological field data?
  \item Are there documented cases of fluctuating equilibrium populations?
\end{itemize}

\textbf{Simulation Viability:}
\begin{itemize}
  \item Could platforms like MESA or NetLogo be used to simulate this model?
  \item What simulation parameters would best reflect real-world variability?
\end{itemize}

\section*{Conclusion}

This theorem introduces a more ecologically realistic view of predator-prey systems by embedding time-dependent environmental changes directly into the carrying capacity term. The implications for ecological modeling, conservation, and resource management are substantial, particularly in the face of climate change and habitat variability.

\end{document}